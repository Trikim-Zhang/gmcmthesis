\section{使用csvsimple宏包生成\LaTeX{}表格}
\subsection{结果}
使用csvsimple 宏包的最简单方式是直接\verb|\csvautotabular|命令生成表格,其代码如下,生成的表格如\cref{tab:csvautotabular}所示。
\begin{table}[htbp]
    \centering
    \caption{使用 csvautotabular 命令生成表格\label{tab:csvautotabular}}
    \csvautotabular{table/name-gender-age.csv}
\end{table}
\subsection{logo}
最后自己用\textsf{Ipe}制作了本文档的logo。
\begin{figure}[htbp]
    \centering
    \includegraphics[width = 2.5in]{image/gmcmthesis.pdf}
    \caption{用\textsf{Ipe}制作的logo}
\end{figure}
\subsection{较长的标题}
% Table generated by Excel2LaTeX from sheet '数据'
\begin{table}[htbp]
    \centering
    \caption{第1、2、4 问的详细定位结果, 每一问的物体数量皆为2个, 给出了物体的直角坐标、物体极坐标、物体的反射系数, 以及利用重建的位置数据生成观测信号与原始观测信号间的$\ell_2$范数误差.}
    \resizebox{\textwidth}{!}{
    \begin{tabular}{ccccc}
        \toprule[1.5pt]
        \multicolumn{1}{c}{数据集} & 物体直角坐标$(x, y)$ & 物体极坐标$(r, \theta)$ & 物体反射系数 & 观测信号的重建误差 \\
        \midrule[1pt]
        \multicolumn{1}{c}{\multirow{2}[2]{*}{1, 无噪声数据}} &   (-0.0183, 7.0048) & (7.0048, -0.0026) & 3.9952 + 3.0058i                  & \multirow{2}[2]{*}{0.0454} \\
            & (0.0184, 7.0048) & (7.0048, 0.0026) & -2.9940 - 4.0046i &  \\
        \hline
        \multicolumn{1}{c}{\multirow{2}[2]{*}{2, 含噪声数据}} &   (-0.0447, 8.2056) & (8.2057, -0.0054) & 4.3719 + 2.0751i               & \multirow{2}[2]{*}{209.4977} \\
            & (0.0447, 8.2056) & (8.2058, 0.0054) & -3.8427 - 2.9398i &  \\
        \hline
        \multicolumn{1}{c}{\multirow{2}[2]{*}{4, 天线老化数据}} &    (0.0159, 6.1042) & (6.1042, 0.0026) & -2.5414 - 4.2246i           & \multirow{2}[2]{*}{169.0732} \\
            & (-0.0158, 6.0041) & (6.0041, -0.0026) & 3.8456 + 3.0986i &  \\
        \bottomrule[1.5pt]
    \end{tabular}%
    }
    \label{tab:data}%
\end{table}%